\documentclass{article}
\usepackage{amsmath}
\usepackage{bm}
\usepackage{graphicx}


\title{FLCKS Restaurant Booking System Architecture Document }


\author{
	Christopher Mashele:\\
	\and
	732475\\
	}
\begin{document}

\begin{titlepage}
	
		\hspace{0.5cm}
		\huge
		\center
		\textbf{COMS3008:\\ Software Design}\\

		\vspace*{1cm}
		\Large
		\hspace{2cm}
		\center
		\textbf{FLCKS Restaurant Booking System Architecture Document}
		\vspace*{2cm}
		\normalsize
		\hspace{2.5cm}
		%\textbf{22 July 2016 14h00}

		\textbf{\underline{Team Members}}\\

		\textbf{Mthulisi Leslie Zimba:\hfill570937}

		\textbf{Kgopotso Dilebo:\hfill 715636}

		\textbf{Lethabo Nkabinde:\hfill 722211}

		\textbf{Christopher Mashele:\hfill 732475}
		
		\textbf{Fortune Ndlovu:\hfill 731603}

		
		\vspace{2cm}
		\normalsize
		\hspace{4cm}
		\includegraphics[scale=0.3]{logo}\\
		\hspace{3cm} School of Computer Science and Applied Mathematics\\
		\hspace{3cm}University of the Witwatersrand\\
		\hspace{3cm} South Africa\\
		\hspace{3cm}  10 October 2016 \\
	
\end{titlepage}

\tableofcontents
\pagebreak
\section{\underline{Introduction}}
\subsection{Scope}
The following document  pertains to, or describes the architecture of a mobile restaurant booking system application created by FLCKZ,  which works as a booking system for restaurants . The application gives you access to restaurant menus, allows one to book tables and order meals,also check table availability and also rate/share about the service received from the restaurant.\\
Certain topics will be discussed at different levels of the Architecture, being relevant to different parties, details to which parties are applicable will be clearly stated.\\\\ The system works on android devices and is downloadable via the Google Play Store on your mobile device.\\

\subsection{{Acronyms and Abbreviations}}
Some Acronyms and Abbreviations used in this document are mentioned below:
\begin{itemize}
\item FLCKZ: Design team name derived from the names of the group Members (Fortune, Lethabo,Chris,Kgopotsho, Mthulisi (Zimba))
\item GUI: Graphic User Interface
\item UML: Unifed Modeling Language

\end{itemize}
\subsection{{Definitions}}
Some Definitions explained:
\begin{itemize}
\item PHP: A server-side scripting language designed for web development.
\item Event-Driven Architecture: An Architecture development design Technique
\item Event-mediator: The event mediator receives the initial event and
orchestrates that event by sending additional asynchronous events
to event channels to execute each step of the process.
\item Event-process: which listen on the event channels, receive the event from the
event mediator and execute specific business logic to process the
event.


\end{itemize}

\subsection{References}
%Here a list of references used in the creation of this Architecture Document:\\\\
%1. D. Garlan & M. Shaw, “An Introduction to Software Architecture,” Advances in Software
%Engineering and Knowledge Engineering, Vol. 1, World Scientific Publishing Co. (1993).
%2. D. E. Perry & A. L. Wolf, “Foundations for the Study of Software Architecture,” ACM Software
%Engineering Notes, 17, 4, October 1992, 40-52.
%3. Ph. Kruchten & Ch. Thompson, “An Object-Oriented, Distributed Architecture for Large Scale Ada
%Systems,” Proceedings of the TRI-Ada ’94 Conference, Baltimore, November 6-11, 1994, ACM,
%p.262-271.
%4. G. Booch: Object-Oriented Analysis and Design with Applications, 2nd. edition, Benjamin-Cummings
%Pub. Co., Redwood City, California, 1993, 589p.
%5. K. P. Birman, and R. Van Renesse, Reliable Distributed Computing with the Isis Toolkit, IEEE
%Computer Society Press, Los Alamitos CA, 1994.
%6. K. Rubin & A. Goldberg, “Object Behavior Analysis,” CACM, 35, 9 (Sept. 1992) 48-62
%7. B. I. Witt, F. T. Baker and E. W. Merritt, Software Architecture and Design—Principles, Models, and
%Methods, Van Nostrand Reinhold, New-York (1994) 324p.
%8. D. Garlan (ed.), Proceedings of the First Internal Workshop on Architectures for Software Systems,
%CMU-CS-TR-95-151, CMU, Pittsburgh, 1995.
 
\pagebreak

\section{Architectural Representation}
This document details the Architecture using the views defined in the "4+1" model, but using the RUP naming convention. The FLCKS application will be using the following views:
\begin{itemize}
\item \textbf{\underline{Logical View}}\\\\ \underline{Audience:} Designers\\ \underline{Area:} Functional Requirements: discribes the design object model. Also describes the most important use-case realizations.\\ \underline{Related Artifact:} Design Model

\item \textbf{\underline{Process View}}\\\\ \underline{Audience:} Integrators\\ \underline{Area:} Non-Functional Requirements: discribes the concurrency and synchronization aspects.\\ \underline{Related Artifact:} None


\item \textbf{\underline{Implementation View:}}\\\\ \underline{Audience:} Programmers\\ \underline{Area:} Software Components: Describes the layers and subsystems of the application\\ \underline{Related Artifact:} Implementation Model.

\item \textbf{\underline{Deployment View:}}\\\\ \underline{Audience:} Deployment Managers\\ \underline{Area:} Topology: discribes the mapping of the software onto hardware and shows system distributed aspects.\\ \underline{Related Artifact:} Deployment Model.

\item \textbf{\underline{Use-Case View:}}\\\\ \underline{Audience:} Stakeholders\\ \underline{Area:} Scenerios and use cases\\ \underline{Related Artifact:} Use-case Model and Use-case documents.

\end{itemize} 
\section{\underline{Architectural goals and constraints}}
This document aims to show how our application will work and to form a blueprint which will enable one to navigate and explore easily through the Application to give one a better idea as to what our Application does.\\ This section describes the software requirements and objectives that have a significant impact on the architecture.

\subsection{Software Decisions}
The System that we are creating is intended to perform well in an ever-changing environment because it is intended on being used on a daily basis, keeping in mind that it also needs to be very secure and thus we wouldn't want any data to be leaked/published, because confidential information will be dealt with. We also want a high performing application. We expect the application to be accessed by a large number of people and thus would like to be able to upgrade/ scale up if needs be in order to accommodate large volumes of users in the future.\\\\ With that in mind we have decided to use the event-driven architectural style because it performs well in the key factors required for the application. We will be using the Mediator Topology in the creation of the Application due to a number of different factors, which will be explained throughout the document.\\\\ Our Application will have two main users, namely the Client and the Administrator, thus we found it fitting that since the Administrator will be at work most of the time, that they should access their Administrative privileges using their office computer. Thus no cost will be put to the company in terms of mapping the software to hardware, they just use their current Office PC.\\ The Client on the other Hand will only be able to access the system via an android device. This decision was made due to recent statistics that show the ever increasing number of android devices available and the number of people who use android devices today.\\\\ Depicted is the Architecture Decision diagram, where the black blocks represent the type of functionality and the box below depicts the domain to which the functionality will be accessed.

\subsection{Technical Platform}

The FLCKZ App will be deployed onto a Firebase server since it was the best platform available to us.

\subsection{Transaction}
The FLCKZ App is transactional, leveraging the technical platform capabilities.
Transaction management model of the lamp will be reused intensively.

\subsection{Security}
The system must be secured, so that a customer can view events on the App.
The application must implement basic security behaviours:

\begin{itemize}
\item  \underline{Authentication}: Login using at least a user name and a password
\item \underline{Authorization}: according to their profile, on-line customer must be granted
or not to perform some specific actions, they should have different granularities of access to the App depending on their profile.\\
For internet access, the following requirements are mandatory
\item \underline{Confidentiality}: sensitive data must be encrypted (passwords)
\item \underline{Data integrity}: Data sent across the network cannot be modified by a
tier
\item \underline{Auditing}: Every sensitive action can be logged
\item \underline{Non-repudiation}: gives evidence a specific action occurred
\end{itemize}

\subsection{Persistence}
Data persistence will be addressed using a relational database.

\subsection{Performance}
The login process must take no longer than 5 seconds.

\subsection{Reliability and Availability}
The availability of the system is a key requirement by nature, as it is a selling
system. The candidate architecture must ensure failover capabilities. Targeted
availability is 12/5: 12 hours a day, 5 days a week The time left (8 hours) is
reserved for any maintenance activities.


\pagebreak


\pagebreak

\pagebreak
\section{\underline{Scenarios}}
\subsection{{Types of Scenarios}}
True there are many different scenarios that could take place, as mentioned in the topic discussed above, and not all can be mentioned in this documentation. But a brief overview of the most basic and common scenarios that will be considered, due to the nature of the system as a whole. Here is a list of the top scenarios that will be fulfilled by the System:
\begin{itemize}
\item Register: This scenario is the first experience you have with the Application, it enables you to create your user profile and adding your details to the local host database system.
\item Login: Once you have a profile, this scenario allows you to access the Application and gives a client the ability to start performing the different tasks given to them by the Application. 
\item Logout: Once you have completed your given tasks, then it allows you to exit your profile and login another profile.
\item View menu: This scenario will be one of the most common because if you dont want to book a table, you can still use this scenario in order to keep up to date with the current meals and specials offered by the restaurant.
\item Book table: This is also a common scenario and was mentioned in detail in the above section.
\item Order meal: This can be done by selecting the meals you fancy in the menu and clicking requesting it by clicking on the order button. 
\end{itemize}

\subsection{Booking a Table and Order Meals}
A customer accesses the FLCKZ App and requests to book a table in a given restaurant. The customer then specifies the occasion and the amount of people need to be accommodated on that table.Then, the customer confirms his booking. After that, if the table is available then a notification is sent back to the customer. And thus the customer will be given the option to order a meal.\\ This is depicted in the following use case diagram\\

\includegraphics[scale=0.5]{Use_Case}
\paragraph{•}
These are some of the functionalities we are working on:
\begin{itemize}
\item Rate service: This will enable you to rate the service you recieved both by the restaurant and the Application in store. More details will revealed as the progress of the Application unfolds.
\item Share experience on social media: Brag to your friends on your experience with the Application and restaurant.
\end{itemize}

\subsection{Use-Case Realization}
Refers to how design elements provide the functionalities
identified in the significant use-cases.
\pagebreak
\section{\underline{Logical View}}
\subsection{Overview}
The FLCKZ application is divided into layers based on the N-tier architecture.
The layering model of the FLCKZ application is based on a responsibility
layering strategy that associates each layer with a particular responsibility. This
strategy has been chosen because it isolates various system responsibilities from
one another, so that it improves both system development and maintenance.

\includegraphics[scale=0.5]{Logical_View} 

\subsection{Architecture Layer Dependency}
\includegraphics[scale=0.5]{Layer_Dependency}\\\\ Each layer has specific responsibilities.
\begin{itemize}
\item  The presentation layer deals with the presentation logic and the pages
rendering
\item The control layer manages the access to the domain layer
\item The resource layer (integration layer) is responsible for the access to the
enterprise information system (databases or other sources of information)
\item The domain layer is related to the business logic and manages the accesses
to the resource layer.
\item The Common Elements layer gathers the common objects reused through
all the layers
\end{itemize}


\pagebreak

\section{\underline{Process View}}
The following diagram describes the state of the program when the user decides to book a table. The system will be in idle until the an occasion is selected. Then if the action is unsuccessful then the
program will move to the final state. Otherwise the program will exit normally
to the select the quantity of people - when and what time the table is booked for. Then on confirmation, the state will change to order confirmation. Then on to confirm order. After all that is done,
the program will move to the final state.\\

\includegraphics[scale=0.9]{Process_View}


\pagebreak
\section{\underline{Physical View }}
The application is Android based. It is assumed to register it's users using a
database stored on the lamp server. So the application would need to connect
to the internet first, then be directed to web servers that are going to grant it
access to the lamp server. Then from the lamp server the application will access
the database. Redirect everything that it got from those databases back to the
web services then back to the application.\\

\includegraphics[scale=0.5]{Physical_View}

\section{\underline{Deployment View }}
\subsection{Layers} 
\subsubsection{Presentation Layer}
The Presentation layer contains all the components needed to allow interactions
with an end-user. It encompasses the user interface
\subsubsection{Control Layer}
The Control layer contains all the components used to access the domain layer
or directly the resource layer when this is appropriate.
\subsubsection{Resource Layer}
The Resource layer contains the components needed to enable communication
between the business tier and the enterprise information systems (Database,
external services, ERP, etc. . . )
\subsubsection{Domain layer}
The Domain layer contains all the components related to the business logic. It
gathers all the subsystems that meet the needs of a particular business domain.
It also contains the business object model.

\subsubsection{Common Elements Layer}
The Common Element layer contains the components re-used within several
layers.\\\\
The Development view depicts the physical composition of the implementation
in terms of Implementation Subsystems, and Implementation Elements (directories and files, including source code, data, and executable files). Usually, the
layers of the Development view do fit the layering defined in the Logical view
It is unnecessary to document the Development view in great details in this
document. For further information, refer to the On-line Catering Service 1.0
workspace in Rational Software Architect.\\

\includegraphics[scale=0.5]{Deployment_View}
%\subsection{{Design Principles}}

\end{document}
